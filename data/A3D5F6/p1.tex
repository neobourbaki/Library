\begin{document}

\begin{proof}
By definition of the derivative, we need to prove that 
$$
\lim_{h\to 0} \left(\frac{1}{h}\int_x^{x+h} f(t)\,dt) = f(x)
$$

Indeed, by continuity of $f(t)$ (at fixed $x$), for 
any $\epsilon>0$ we may choose $\delta$ such that
$$
f(x)-\epsilon<f(t)<f(x)+\epsilon
$$
for $|t-x|<\delta$. Integrating $f(t)$ from $x$ to $x+\delta$, we have
$$
(f(x)-\epsilon)\, \delta < \int_x^{x+\delta} f(t)\,dt < (f(x)+\epsilon)\, \delta
$$
, which implies that
$$
\left| \frac{1}{\delta}\int_x^{x+\delta} f(t)\,dt - f(x) \right| < \epsilon
$$
(and similarly for the integral to $x-\delta$). That is, the same $\epsilon$ establishes 
the claimed limit.

\end{proof}

\end{document}