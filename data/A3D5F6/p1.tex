\begin{document}

\begin{proof}
By definition of the derivative, we need to prove that 
$$
\lim_{h\to 0} \frac{1}{h}\int_x^{x+h} f(t)\,dt = f(x)
$$

Indeed, by continuity of $f(t)$ (at fixed $x$), for 
any $\epsilon>0$ we may choose $\delta$ such that
$$
f(x)-\epsilon<f(t)<f(x)+\epsilon
$$
for $|t-x|<\delta$. Integrating $f(t)$ from $x$ to $x+h$, $0<h<\delta$, we have
$$
(f(x)-\epsilon)\, h < \int_x^{x+h} f(t)\,dt < (f(x)+\epsilon)\, h
$$
which implies that
$$
\left| \frac{1}{h}\int_x^{x+h} f(t)\,dt - f(x) \right| < \epsilon
$$
(and similarly for the integral from $x$ to $x-h$). That is, the same 
$\delta$ establishes the claimed limit.

\end{proof}

\end{document}