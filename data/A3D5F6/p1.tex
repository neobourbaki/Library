\begin{document}

\begin{proof}
By definition of the derivative, we need to prove that 
$$
\int_x^{x+h} f(t)\,dt = f(x)\, h + o(h)
$$
as $h\to 0$. Indeed, by continuity of $f(t)$ (at $t=x$), for 
any $\epsilon>0$ we may choose $\delta$ such that
$$
f(x)-\epsilon<f(t)<f(x)+\epsilon
$$
for $|t-x|<\delta$. Integrating $f(t)$ from $x$ to $x+h$, with 
$0<h<\delta$, we have
$$
(f(x)-\epsilon)\, h < \int_x^{x+h} f(t)\,dt < (f(x)+\epsilon)\, h
$$
(and similarly for $-\delta<h<0$), which implies that 
$$
\int_x^{x+h} f(t)\,dt = f(x)\,h + o(h)
$$

\end{proof}

\end{document}