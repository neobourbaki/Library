\begin{document}

\begin{proof} Assume in addition that $f'$ is continuous.
With the Fundamental Theorem of Calculus,
$$
f(b)-f(a) = \int_a^b f'(x)\,dx = f'(\xi)\,(b-a)
$$
for some $\xi$, as a consequence of the intermediate 
value theorem. Indeed, unless $f'$ is constant, in which case 
$$
f'(\xi)=\frac{1}{b-a}\int_a^b f'(t)\,dt
$$
for all $\xi$, it must assume a value more, and also another value less, 
than this mean value, somewhere on the interval $(a,b)$.


\end{proof}

\end{document}