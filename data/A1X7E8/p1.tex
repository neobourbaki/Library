\begin{document}

\begin{proof}
The case that $f(a)=f(b)$ (known as Rolle's Theorem) 
is a consequence of the extreme value theorem. Indeed, $f$ must attain 
its maximum and minimum, one of which must be in the interior of $[a,b]$. 
Since $f$ is differentiable there, the derivative must be zero.

In general, let $L:[a,b]\to\RR$ be the linear function 
with $L(a)=f(a)$ and $L(b)=f(b)$, i.e.,
$$
L(x):=f(a) + \frac{f(b)-f(a)}{b-a}(x-a)
$$
Then the difference $h(x):=f(x)-L(x)$ satisfies the conditions 
of Rolle's Theorem, thus there exists $\xi\in (a,b)$ such 
that 
$$
h'(\xi) = f'(\xi)-\frac{f(b)-f(a)}{b-a} = 0
$$
as desired.
\end{proof}

\end{document}