\begin{document}

\begin{proof}

Let 
$$
g(x) := f(x) - \int_a^x f'(t)\,dt 
$$
as a function of $x\in I$, with $a$ fixed. It is enough to 
show that $g$ is constant on $I$, for then 
$$
g(b) = g(a) = f(a) - 0.
$$

Indeed, by another version of the Fundamental Theorem of Calculus,
$$
g'(x) = f'(x) - \frac{d}{dx}\int_a^x f'(t)\,dt = f'(x) - f'(x) = 0
$$
at any $x\in I$. Now, let $c<d$ be any two points of $I$, and the 
mean value theorem implies that there exists $\xi\in (c,d)$ such that
$$
g(d) - g(c) = g'(\xi)\, (d-c) = 0
$$
which means that $g$ is constant on $I$.

\end{proof}

\end{document}