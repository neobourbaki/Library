\begin{document}

\begin{proof}

Let 
$$
F(b) := f(b) - \int_a^b f'(x)\,dx 
$$
as a function on $I$, with $a$ fixed. It is enough to 
show that $F_a$ is constant on $I$, for then 
$$
F(b) = F(a) = f(a).
$$

Indeed, by another version of the Fundamental Theorem of Calculus,
$$
F'(b) = f'(b) - \frac{d}{db}\int_a^b f'(x)\, dx = f'(b) - f'(b) = 0
$$
at any $b$ in the interior of $I$, while $F$ is continuous on all $I$. 
Now, let $c<d$ be any two points of $I$, and the mean value theorem 
implies that there exists $\xi\in (c,d)$ such that
$$
F(d) - F(c) = F'(\xi)\, (d-c) = 0
$$
which means that $F$ is constant on $I$.

\end{proof}

\end{document}