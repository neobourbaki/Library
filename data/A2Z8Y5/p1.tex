\begin{document}

\begin{proof}
If $f$ is constant, then  
$$
\int_a^b f(x)\,dx = f(\xi)\,(b-a) 
$$
for all $\xi$. Now assume $f$ is not constant, then $f$ must assume 
a value more, and another value less, than this mean value, 
$$
\frac{1}{b-a}\int_a^b f(x)\,dx
$$
somewhere on the interval $(a,b)$. By the intermediate value theorem, 
$f$ must assume the mean value in between these two points.

\end{proof}

\end{document}